\documentclass[a4paper,12pt]{article} % Paper size and font size

% Margins
\usepackage[a4paper, margin=1in]{geometry}

% Font encoding
\usepackage[T1]{fontenc}

% Language
\usepackage[english]{babel}
\usepackage{float}
% Line spacing
% \usepackage{setspace}
% \onehalfspacing % Adjust this as needed

% Header and footer
\usepackage{fancyhdr}
\pagestyle{fancy}
\fancyhf{}
\fancyhead[L]{\textbf{Results}} % Left header
\fancyhead[R]{\thepage} % Right header with date
% \fancyfoot[C]{\thepage} % Center footer with page number

% Bullet points and lists
\usepackage{enumitem}

% Additional useful packages
\usepackage{amsmath} % For mathematical typesetting
\usepackage{amsfonts} % For more fonts
\usepackage{amssymb} % For additional symbols
\usepackage[style=apa]{biblatex}

% No indentation for paragraphs
\setlength{\parindent}{0pt}

% Spacing between paragraphs
\setlength{\parskip}{0.5em} % Adjust this as needed

% Page number style
\renewcommand{\thepage}{\arabic{page}}

\addbibresource{himalayan-bib.bib}

\begin{document}

% Title or introductory text
\title{The Himalayan Tightrope: A High-resolution Study of the Economy-Environmental Trade-off of Infrastructure Development \\[1em]\textit{Results}}
% \author{Ritwiz Sarma\\Riju Garg}
\date{}
\maketitle

\section{Recap: Data}

Our data is as follows:

\begin{description}
    \item[Night lights (NTL)] Sourced from \textcite{Li2020ntl}. Spatial resolution: 1km. Temporal range: 1992 -- 2023.
    \item[Built-up area (GHSL)] Sourced from Global Human Settlements Layer, Copernicus (ESA). Spatial resolution: 0.1km. Temporal range: 1975 -- 2030 in 5-year gaps.
    \item[Land surface temperature (LST)] Sourced from MOD11A2, Terra MODIS (NASA). Spatial resolution 1km. Temporal range: 2000 -- 2020.
    \item[Forest cover (VCF)] or Vegetation Continuous Fields. Sourced from DevDataLab cf. MOD44B, Terra MODIS (NASA). Spatial resolution: 0.25km. Temporal range: 2001 -- 2020. 
    \item[PM2.5 air pollution (PM2.5)] Sourced from DevDataLab cf. \textcite{van2021pm25}. Spatial resolution: 1km. Temporal range: 2000 -- 2020.
\end{description}

\section{Indexation}

We have two methods of indexation: \textit{uniform weighting} (UW) or \textit{principal component analysis} (PCA). Principal component analysis is helpful in the presence of high multicollinearity. To check for high correlation between variables, i.e., to understand if PCA is required, we undertake two tests: the \textit{Kaiser–Meyer–Olkin (KMO) test} and \textit{Bartlett's test of sphericity}.

\begin{table}[H]
\centering\begin{tabular}{l|r|r}
    Variables       &   KMO MSA         &   Bartlett p-value        \\
    \hline 
    VCF, PM2.5      &   0.5, 0.5        &   0                       \\
    VCF, PM2.5, LST &   0.68, 0.57, 0.56&   0                       \\
    

\end{tabular}
\end{table}

\section{Models}

\begin{equation}
    \rm{NTL} = \underbrace{\rm{VCF} + \rm{PM2.5} + \rm{LST}}_{\text{PCA required}}
\label{eqn:model1}
\end{equation}

Our first model has the highest $T$ (i.e., 20), and is thus more efficient for estimating our epochs. We can consider showing only this for epoch analysis, considering that $\text{GHSL}$ is correlated with NTL anyway. I am obliged to estimate this both with and without the LST term, owing to the justification we have provided for Model~\ref{eqn:model2}.


\begin{equation}
    \underbrace{\rm{NTL} + \rm{GHSL}}_{\text{Both PCA and UW}} = \underbrace{\rm{VCF} + \rm{PM2.5}}_{\text{Both PCA and UW}}
\label{eqn:model2}
\end{equation}

Our second model is motivated by LST having more long-run effects than the other environmental indicators. It is limited by having $T = 5$, which necessarily makes the estimation of one epoch an FD instead of FE model.


\begin{equation}
    \underbrace{\rm{NTL} + \rm{GHSL}}_{\text{Both PCA and UW}} = \underbrace{\rm{VCF} + \rm{PM2.5} + \rm{LST}}_{\text{PCA required}}
\label{eqn:model3}
\end{equation}

This is the unrestricted model with all indicators. Also has $T = 5$.


\section{Results}

% Table created by stargazer v.5.2.3 by Marek Hlavac, Social Policy Institute. E-mail: marek.hlavac at gmail.com
% Date and time: Sun, Jan 05, 2025 - 23:55:09
\begin{table}[hbt] \centering 
  \caption{Regression results from Model 1 for Epoch 1.} 
  \label{} 
\begin{tabular}{@{\extracolsep{5pt}}lccc} 
\\[-1.8ex]\hline 
\hline \\[-1.8ex] 
 & \multicolumn{3}{c}{\textit{Dependent variable:}} \\ 
\cline{2-4} 
\\[-1.8ex] & ntl\_mean\_st & ntl\_mean & ntl\_mean\_st \\ 
\\[-1.8ex] & (1) & (2) & (3)\\ 
\hline \\[-1.8ex] 
 env\_index & 0.144$^{***}$ &  &  \\ 
  & (0.00001) &  &  \\ 
  & & & \\ 
 env\_index\_sq & 0.068$^{***}$ &  &  \\ 
  & (0.00000) &  &  \\ 
  & & & \\ 
 VCF\_LST\_PC &  & 0.434$^{***}$ &  \\ 
  &  & (0.0004) &  \\ 
  & & & \\ 
 VCF\_LST\_PC$^2$ &  & 0.567$^{***}$ &  \\ 
  &  & (0.0001) &  \\ 
  & & & \\ 
 VCF\_LST\_UW &  &  & 0.097$^{***}$ \\ 
  &  &  & (0.00002) \\ 
  & & & \\ 
 VCF\_LST\_UW$^2$ &  &  & 0.179$^{***}$ \\ 
  &  &  & (0.00000) \\ 
  & & & \\ 
\hline \\[-1.8ex] 
Observations & 97,758 & 97,875 & 97,875 \\ 
R$^{2}$ & 0.120 & 0.101 & 0.101 \\ 
\hline 
\hline \\[-1.8ex] 
\textit{Note:}  & \multicolumn{3}{r}{$^{*}$p$<$0.1; $^{**}$p$<$0.05; $^{***}$p$<$0.01} \\ 
\end{tabular} 
\end{table} 


\end{document}
